\documentclass[conference]{IEEEtran}
%\IEEEoverridecommandlockouts
% The preceding line is only needed to identify funding in the first footnote. If that is unneeded, please comment it out.
\usepackage{cite}
\usepackage{amsmath,amssymb,amsfonts}
\usepackage{algorithmic}
\usepackage{graphicx}
\usepackage{textcomp}
\usepackage{xcolor}

\begin{document}

\title{Lifelong Planning: A Comparative Evaluation of Route Replanning In Partially-Observable Domains}

\author{\IEEEauthorblockN{Hans Walter Behrens, Bharath Gunari, Rishabh Hatgadkar, Nicholas Martinez (1207024399)}
\IEEEauthorblockA{\textit{Arizona State University}\\
	Tempe, Arizona 85281--3673\\
\texttt{\{hwb, bgunari, rhatgadk, nlmarti4\}@asu.edu}}
%\and
%\IEEEauthorblockN{Bharath Gunari}
%\IEEEauthorblockA{\textit{Arizona State University} \\
%\texttt{@asu.edu}}
%\and
%\IEEEauthorblockN{Rishabh Hatgadkar}
%\IEEEauthorblockA{\textit{Arizona State University} \\
%\texttt{@asu.edu}}
%\and
%\IEEEauthorblockN{Nicholas Martinez}
%\IEEEauthorblockA{\textit{Arizona State University} \\
%\texttt{@asu.edu}}
}

\maketitle

\begin{abstract}
Route-finding has historically played an important role within computer science, but often assumes that the region or graph to be traversed is fully known. In practice, this requirement can prove difficult to satisfy; no domain better illustrates this challenge than robotics. As autonomous agents traverse unknown environments, their route-finding must adapt to incorporate new information as it becomes available. In this work, we explore several route-finding algorithms from the literature, including lifelong-planning A*, D* Lite, and naive-replanning A*. We discuss the implementations of these approaches, and discuss their strengths, weaknesses, and appropriateness for various scenarios. We conclude by comparing their relative efficacy for autonomous agent pathfinding in partially-observable terrain, as represented by the Pacman pathfinding problem.
\end{abstract}

\begin{IEEEkeywords}
path planning, shortest path problem, robot learning, observability
\end{IEEEkeywords}

% Introduction (your goal); 

\section{Introduction}

%Background (a brief technical discussion of the D* Lite search method you implemented); figures may be helpful here; 


\section{Background}
	This section will contain a brief technical discussion and intuition for each of the three approaches.
	
	\subsection{Naive Re-planning A*}
	    The NRA* algorithm is an adaption of the classic A* Search to an environment that is partially observable. In an A* Search, a rational agent prioritizes nodes $s$ in the fringe based on a value $f(s) = g(s)+h(s)$, which corresponds to the cost to reach the current node, plus the heuristic cost to reach the goal. In NRA*, the only information that the agent can perceive are the four adjacent cells in the assumed grid-world environment. Therefore, the NRA* agent generates an optimal path, and follows it until one of two things occur: (a) it has reached the goal or (b) it encounters a previously-unobserved obstacle. In this second case, it updates the edge weights into the obstacle square, re-runs A* search from its current location, and repeats the process as before. N.B. that when this re-planning takes place, any previous route knowledge that might have been computed previously does not carry over to the next search, resulting in significant recomputation.

	\subsection{Lifelong Planning A*}
	    The LPA* search algorithm is based on an incremental heuristic search \cite{koenig2002d} \cite{koenig2002improved}. Unlike NRA*, LPA* is able to use information from previous routefinding computations to inform new searches. To do so, LPA* keeps track of $g(s), rhs(s)$ 2-tuples for each node $s$ in the graph. Intuitively, $g(s)$ is an estimate of the cost so far, and corresponds to the same value from A* search. $rhs(s)$ is a one-step look-ahead estimate that takes into consideration the predecessors $s'$ of $s$, potentially providing a better estimate than $g(s)$. More formally, the $rhs$ is defined as 0 if $s=s_{start}$ and $min_{s'\in pred(s)} g(s')+c(s',s)$ otherwise. The queuing strategy for fringe nodes is more complex than in (NR)A*, as it uses a tuple of keys that establish a dual-priority system. Since the $rhs(s)$ values are a ``potentially better informed'' version of the $g(s)$ values, a node is called locally inconsistent if $g(s) \neq rhs(s)$. When a node becomes locally inconsistent, either through initialization or replanning, it is added to this queue.
	    
	    Initially, only the start state is locally inconsistent. By making the nodes locally consistent, and therefore making sure that the $rhs$ values are precisely the same as the $g$ values, an optimal path can be found based on the currently-known edge weights. Upon discovering environmental changes which affect the edge weights between nodes, it updates those affected nodes, and makes them locally inconsistent (or in other words, adds them to the priority queue) if necessary. It then repeats the pathfinding process until the graph is once again consistent. Note that edge weights can increase (locally overconsistent) or decrease (locally underconsistent) in general, but in our example scenario, edges can only increase. This corresponds to the agent encountering an impassable barrier. Once the graph is locally consistent, the shortest path to the goal can be found by starting at the goal state, transitioning back to the predecessor through the nodes which minimize the cost, continuing until the start state is reached.  
	
		A key drawback for LPA* is that it does not allow for changing start positions. For our motivational scenario, where the agent must directly observe the updated edge weights, this can cause significant backtracking.
	
	\subsection{D* Lite}
		 D* Lite \cite{koenig2004lifelong} \cite{simmons1995probabilistic} aims to improve on this by only updating nodes on the path from the in-progress current position to the goal state. To do so, modest changes to the LPA* algorithm are required; however, the core replanning approach of LPA* is maintained.
		
		LPA* iteratively attempts to find shortest paths from the start to the goal, incorporating new edge costs as it observes them through the local consistency approach described above. In contrast, D* Lite iteratively attempts to find the shortest path from the $current$ node to the goal, also incorporating new edge costs as it observes them, but only towards the goal. There are some other differences as well. In particular, the search direction in now switched, and the $g(s)$ values now represent the goal distance, not the backwards cost. Similarly, the $rhs$ values are now forward looking, i.e. one step look-ahead with respect to successors and their respective goal distances. Similar to LPA*, D* Lite also prioritizes nodes based on a tuple that is defined in terms of  $g(s), rhs(s)$. However, the definition of this tuple differs from that of LPA* in that its priorities are effectively lower bounds on those priorities from LPA*. It also maintains a priority offset $k_m$, which allows it to bound the replanning propagation. This re-formulation, along with a less strict terminating condition allow D* Lite to have better performance than LPA*.
		
		The framework for D* Lite is as follows: it first attempts to compute the shortest path from the start to the goal node. It then takes steps along this path, and makes any changes to the edge costs and associated priority queue values when necessary. It then computes the shortest path using this new information, and repeats. The currently-optimal shortest path to the goal can be found by starting at the current start state, transitioning to the successor which minimizes the $rhs$ condition, i.e. one step look-ahead, and continuing until the goal state is reached. However, to extract the actual path taken by the agent including replanning, this information must be extracted retrospectively, by recording the steps taken.

% Implementation details for the path finding problemsas the Pacman plans and replans in the environment; figures are helpful here. 

\section{Implementation Details}\label{sec:implementation}

	All implementation was done in Python2, for ease of integration with other systems. 

\subsection{Pacman}

	A common context for measuring pathfinding and route-planning algorithms is through the classic arcade game, Pacman. By adjusting different aspects of gameplay, such as the presence of ghosts, fore-knowledge of wall locations, density and placement of food pellets, and so on, a wide variety of heterogeneous environmental configurations can be used to explore different avenues of route planning.
	
	

\subsection{Naïve-Replanning A*}

	Naïve-Replanning A* (or NRA*) describes a strategy where the agent first generates a shortest-path estimate to the desired goal using the well-known A* algorithm \cite{hart1968formal}, and then follows that path. As the agent moves through the partially-observable environment, if the route is interrupted by any impassable barriers (such as walls), it updates an internal representation of the environment with that information. It then generates a new shortest path from scratch using A*, using its current location as the start location, the original goal as the new goal, and the updated environmental representation as a modified graph.
	
	This approach is called ``naïve" because it does not re-use any of the previous planning computations for later replanning, which can lead to re-exploration of many states, and unnecessarily duplicated computation. However, it will provide an optimal solution, as A* is guaranteed to find the best route given the known environmental information.
	
	These characteristics (optimal routes, inefficient computation) provided an effective baseline for our comparison, allowing differences between possible routes found through other approaches to be easily detected, although no such differences were expected due to the guarantees provided by the other approaches. Additionally, it established an upper-limit on the number of nodes expanded, since no exploration is reused. In conjunction with the lower-limit established by an omniscient agent using a single, optimal A* search, this permitted a theoretical bounding for the other approaches under consideration.

\subsection{Lifelong-Planning A*}
	Lifelong Planning A* was implemented by by focusing on a core implementation, then by making small changes to configure it with the Pacman domain. This was accomplished using object oriented design principles. In particular, the LPA* object must be instantiated with the maze dimensions, the start location, and goal location. It maintains a list of so-called naive walls, which are the walls the agent knows exist up to that current point in time. There is also a mechanism with which the agent can ``make'' a wall in the naive walls so that its knowledge about the maze can persist. The group essentially implemented a wrapper around this LPA* object: initially supplying information, iteratively extracting path steps from the object, and supplying information about the true walls to the object. Since the LPA* agent can backtrack, i.e. revisit parts of the maze multiple times, care had to be taken to make sure that this backtracking behavior was captured in the final path returned for visualization. Each time new information was supplied to the LPA* object, if this associated update caused any of the edge weights to become locally inconsistent, then the LPA* object would update the specific vertex, and possible generate a new search plan.
	
	One of the challenges that the group faced with this part of the implementation was representing the search process that the Pacman took. Instead of simply returning the final path, if there was a backtracking situation, the backtrack path had to be appended to the existing path without any overlap. Extensive debugging was involved with this functionality.
	

\subsection{D* Lite}
	Similar to the LPA* implementation, the D* Lite algorithm was also implemented in an object oriented manner. The overall search again acted as a wrapper to the D* Lite object, initializing it, and iteratively extracting information from and updating information in the object. However there was a subtle difference in how the D* Lite object operates. With the D* Lite object, we could simply take a step with the object, extract information from the walls, and repeat. The reason that no backtracking had to be taken into account is because the D* Lite algorithm only expands those nodes which will affect the final path, and this avoids the backtrack step.
	
	Another specific challenge that the group faced with both LPA* and D* Lite integration were the initial differences between the core algorithm and Pacman domain. In particular, the core algorithm used a coordinate system that was transposed with respect to the Pacman domain. Additionally, the core implementation returned a list of coordinate points, rather than actions. Both of these discrepancies had to be re-factored in the code. Finally, the Pacman domain maintains a count of nodes expanded, each time a call to \text{get successors} is made. Since our LPA* and D* Lite core did not interact with \text{get successors}, an internal count had to be maintained so that it could be tracked as a metric for evaluation. 

\subsection{Test Coverage}

	To ensure correct execution of the implementations, unit tests were created in conjunction with the PyCoverage tool. These units tests were used to test the correctness of the implementations against a series of inputs and expected outputs for both the underlying algorithmic implementations, as well as the supporting data structures and algorithms used.
	
	Additionally, these tests were useful in performing regression testing, to ensure that changes to the core algorithmic implementations necessary for integration with the Pacman domain did not compromise their underlying logic.
	
	Our test suite provided complete (100\%) coverage for all core implementations and supporting libraries which were implemented in the course of this project. Supporting code, such as that provided through the Pacman project, system libraries, or other external sources was judged as outside the scope of our testing strategy.




% comparison results (number of nodes expanded, time, etc.) for the path finding problemsas the Pacman plans and replans in the environment; figures are helpful here. 


\section{Evaluation}\label{sec:eval}

\section{Conclusion}
In this research, the team explored the intuition, implementation, and evaluation of three search strategies for agents in  partially-observable (specifically eight-connected) environments. These search strategies were integrated with the Pacman domain, which is a representative application for robot navigation search tasks. The metrics for evaluation were the final path length, number of nodes expanded from the search queue, memory use, and runtime. After evaluating the three search strategies (also using A* as a baseline), the group confirmed that the D* Lite strategy is the optimal re-planning strategy. The D* Lite builds off the theoretical foundation of LPA*, but makes signification optimization modifications to the implementation that allow it dominate in terms of performance. The NRA* strategy is competitive in terms of the path length, but its simple strategy hurts its performance in all other categories. When a totally observable environment is not obtainable, D* Lite is the best search strategy to use. 

\section{Team Notes}
Test text here. 


\section*{Acknowledgment}
% better safe than sorry!
We would like to thank Mehrdad Zaker Shahrak and Yu Zhang for sharing their valuable insights in the course of this work.

\bibliographystyle{ieeetr}
\bibliography{refs.bib}

\end{document}
