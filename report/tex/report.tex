\documentclass[conference]{IEEEtran}
%\IEEEoverridecommandlockouts
% The preceding line is only needed to identify funding in the first footnote. If that is unneeded, please comment it out.
\usepackage{cite}
\usepackage{amsmath,amssymb,amsfonts}
\usepackage{algorithmic}
\usepackage{graphicx}
\usepackage{textcomp}
\usepackage{xcolor}

\begin{document}

\title{Lifelong Planning: A Comparative Evaluation of Route Replanning In Partially-Observable Domains}

\author{\IEEEauthorblockN{Hans Walter Behrens, Bharath Gunari, Rishabh Hatgadkar, Nicholas Martinez}
\IEEEauthorblockA{\textit{Arizona State University}\\
	Tempe, Arizona 85281--3673\\
	\{\texttt{1211230537, 1213217679, 1215005506, 1207024399}\} \\
\texttt{\{hwb, bgunari, rhatgadk, nlmarti4\}@asu.edu}}
%\and
%\IEEEauthorblockN{Bharath Gunari}
%\IEEEauthorblockA{\textit{Arizona State University} \\
%\texttt{@asu.edu}}
%\and
%\IEEEauthorblockN{Rishabh Hatgadkar}
%\IEEEauthorblockA{\textit{Arizona State University} \\
%\texttt{@asu.edu}}
%\and
%\IEEEauthorblockN{Nicholas Martinez}
%\IEEEauthorblockA{\textit{Arizona State University} \\
%\texttt{@asu.edu}}
}

\maketitle

\begin{abstract}
Route-finding has historically played an important role within computer science, but often assumes that the region or graph to be traversed is fully known. In practice, this requirement can prove difficult to satisfy; no domain better illustrates this challenge than robotics. As autonomous agents traverse unknown environments, their route-finding must adapt to incorporate new information as it becomes available. In this work, we explore several route-finding algorithms from the literature, including lifelong-planning A*, D* Lite, and naive-replanning A*. We discuss our implementations of these approaches, and assess their strengths, weaknesses, and appropriateness for various scenarios. We conclude by comparing their relative efficacy for autonomous agent pathfinding in partially-observable terrain, as represented by the Pacman pathfinding problem.
\end{abstract}

\begin{IEEEkeywords}
path planning, shortest path problem, robot learning, observability
\end{IEEEkeywords}

% Introduction (your goal); 

\section{Introduction}

%Background (a brief technical discussion of the D* Lite search method you implemented); figures may be helpful here; 


\section{Background}
	This section will contain brief technical discussion and intuition for each of the three searches.
	\subsection{Naive Re-planning A*}
	    The Naive Re-planning A* is an adaption of the classic A* Search to an environment that is not completely observable. In an A* Search, the rational agent prioritizes nodes $s$ in the fringe based on the $g(s)+h(s)$ value, which is the backwards cost of $s$ summed with the heuristic estimation for $s$. In the case of Naive Re-planning A*, the only sensor information that the rational agent has are the eight adjacent cells (we assume a grid world environment). Therefore, when the Naive Re-planning A* agent has a search-generated path, it must follow that path until a) it has reached the goal or b) it encounters an obstacle. What the Naive Re-planning A* agent does when it reaches an obstacle is simply re-run the A* search from that location, then proceed as before. This continues until the goal is reached. Notice that when this re-planning takes place, any knowledge that might have been captured about the environment does not carry over to the next search. This is addressed in the following search algorithms.

	\subsection{Lifelong Planning A*}
	    The Lifelong Planning A* search algorithm is an incremental heuristic search. It is incremental because it uses information from previous search tasks to inform the current search task. Lifelong Planning A* similarly keeps track of the $g(s),h(s)$ values for the nodes, and additionally maintains a list of $rhs(s)$ values for the nodes, which are one-step look-ahead estimates that take into consideration the predecessors $s'$ of $s$. Formally, the $rhs$ is defined as 0 if $s=s_{start}$ and $min_{s'\in pred(s) + c(s,s')}$ otherwise. The queuing strategy for fringe nodes is slightly more complex, as it uses a tuple of keys that utilize the $g(s),h(s),$ and $rhs(s)$ values. Since the $rhs(s)$ values are a ``potentially better informed'' version of the $g(s)$ values, a node is called locally inconsistent $g(s) \neq rhs(s)$. The queue is then filled with those nodes that are locally inconsistent. By making the nodes locally consistent, an optimal path will be found. 
	    The framework for Lifelong Planning A* is as follows: it attempts to compute the shortest path to the goal node. Upon discovering environmental factors which affect the edge weights between nodes, it updates those affected nodes, makes them locally inconsistent, and therefore inserts them into the queue. It continues searching, until the goal state is found. 
	
	\subsection{D* Lite}
	Lifelong Planning A* Search is an incremental search algorithm which can carry information from one search task to the next, and thus is more effective that D* Lite. However, in Lifelong Planning A*, there are so nodes that are updated, but will have no bearing on the final search path. D* Lite aims to improve on the Lifelong Planning A* search by further optimizing it such that, upon learning of new edge costs, only those nodes are updated that will have an affect on the final search path. In order to do that, the D* Lite make some variations on the Lifelong Planning formulation. In particular, the search direction in now switched, and the $g(s)$ values now represent the goal distance, not the backwards cost. Similar to Lifelong Planning, the D* also prioritizes nodes based on a tuple that is defined in terms of  $g(s),h(s),$ and $rhs(s)$. 	


Placeholder references
\cite{koenig2002d}\cite{koenig2002improved}\cite{koenig2004lifelong}\cite{simmons1995probabilistic}

% Implementation details for the path finding problemsas the Pacman plans and replans in the environment; figures are helpful here. 

\section{Implementation Details}\label{sec:implementation}

	All implementation was done in Python2, for ease of integration with existing systems, namely the Pacman domain. 

\subsection{Pacman}

	A common context for measuring pathfinding and route-planning algorithms is through the classic arcade game, Pacman. By adjusting different aspects of gameplay, such as the presence of ghosts, fore-knowledge of wall locations, density and placement of food pellets, and so on, a wide variety of heterogeneous environmental configurations can be used to explore different avenues of route planning.
	
	In this scenario, we assume a grid-world that can be traversed in 4 directions, $\{N, S, E, W\}$, which is surrounded by impassable barriers. The agent is unaware of any additional walls, but can observe walls immediately adjacent to it. Since there may be large numbers of walls in complex configurations, this is roughly analogous to a blind person finding their way through a labyrinth by touch, a daunting task.

\subsection{Naïve-Replanning A*}

	Naïve-Replanning A* describes a strategy where the agent first generates a shortest-path estimate to the desired goal using the well-known A* algorithm \cite{hart1968formal}, and then follows that path. As the agent moves through the partially-observable environment, if (and only if) the route is interrupted by any impassable barriers (such as walls), it updates an internal representation of the environment with that information. It then generates a new shortest path from scratch using A*, using its current location as the start location, the original goal as the new goal, and the updated environmental representation as a modified graph.
	
	This approach is called ``naïve" because it does not re-use any of the previous planning computations for later replanning, which can lead to re-exploration of many states, and unnecessarily duplicated computation. Additionally, although A* is guaranteed to find the best route given the known environmental information, NRA* is not; this is because it operates with missing information, and can lead the agent down a suboptimal path.
	
%	These characteristics provided an effective baseline for our comparison, allowing differences between possible routes found through other approaches to be easily evaluated.

\subsection{Lifelong-Planning A*}
	Lifelong Planning A* was implemented by creating a generalized implementation, then by making small changes to interface it with the Pacman domain. This was accomplished using object-oriented design principles. In particular, the LPA* object must be instantiated with the maze dimensions, the start location, and goal location. It maintains a list of so-called naive walls, which are the walls the agent is certain exist. We have implemented a mechanism with which the agent can ``make'', or learn, a wall in the naive walls so that its knowledge about the maze can be updated and persist. This wraps the LPA* object: initially supplying information, iteratively extracting path steps from the object, and supplying information about the true walls to the object. Since the LPA* agent must backtrack to compensate for the unchanging start position in LPA* (i.e. revisit parts of the maze multiple times), care had to be taken to make sure that this backtracking behavior was captured in the final path returned for visualization. Each time new information was supplied to the LPA* object, edge weights were updated, and the LPA* object would generate a new route automatically if needed.
	
	One of the challenges that the group faced with this part of the implementation was representing the search process that the Pacman took. Instead of simply returning the final path, if there was a backtracking situation, the backtrack path had to be appended to the existing path without any overlap. Rather than forcing the agent to return to the start position in these cases, we instead intersected the inverse paths to the furthest downstream point, reducing extraneous movement if possible. Extensive debugging was required to provide this functionality.
	
\subsection{D* Lite}
	Similar to the LPA* implementation, the D* Lite algorithm was also implemented in an object oriented manner. The overall search again acted as a wrapper to the D* Lite object, initializing it, and iteratively extracting information from and updating information in the object. However there was a subtle difference in how the D* Lite object operates. With the D* Lite object, since the start position corresponded to the agent's current position, we could simply take a step with the object, learn information from the perceived walls, and repeat. Therefore, no backtracking had to be taken into account because the D* Lite algorithm calculates the shortest path from the current node rather than from the start location. 
	
	Another specific challenge that the group faced with both LPA* and D* Lite integration were the initial differences between the core algorithm and Pacman domain. In particular, the core algorithm used a coordinate system that was transposed with respect to the Pacman domain. Additionally, the core implementation returned a list of coordinate points, rather than actions. Both of these discrepancies had to be accounted for in the code. Finally, the Pacman domain maintains a count of nodes expanded, each time a call to \text{get successors} is made. Since our LPA* and D* Lite core did not interact with \text{get successors}, an internal expansion count had to be maintained so that it could be tracked as a metric for evaluation.

\subsection{Test Coverage}

	To ensure correct execution of the implementations, unit tests were created in conjunction with the PyCoverage tool. These units tests were used to test the correctness of the implementations against a series of inputs and expected outputs for both the underlying algorithmic implementations, as well as the supporting data structures and algorithms used.
	
%	Additionally, these tests were useful in performing regression testing, to ensure that changes to the core algorithmic implementations necessary for integration with the Pacman domain did not compromise their underlying logic.
	
	Our test suite provided complete (100\%) coverage for all core implementations and supporting libraries which were implemented in the course of this project. Supporting code, such as that provided through the Pacman project, system libraries, or other external sources was judged as outside the scope of our testing strategy.




% comparison results (number of nodes expanded, time, etc.) for the path finding problemsas the Pacman plans and replans in the environment; figures are helpful here. 


\section{Evaluation}\label{sec:eval}
	\subsection{Metrics}
    The two main metrics used to measure the performance of the three search algorithms were runtime and nodes expanded. Runtime captures the effective amount of time was needed to run the search algorithm and find an optimal path. The runtime was captures by using the linux $\texttt{time}$ command. The other metric nodes expanded, measures the number of nodes that needed to be expanded, i.e. popped from the priority queue in order to complete the search. The nodes expanded metric is linked to both time and space complexity. This is because the more nodes expanded, the longer the search will take. Also, the more nodes expanded, the more space is needed to store those nodes in the queue. Both of these metrics were measured as a function of maze size, i.e. the effective area of a the Pacman maze in test. Note that there is not a direct linear relationship between maze size and search complexity, since the maze size does not capture other factors, such as maze walls. However, the maze size is correlated with search complexity and in general, the larger the maze size, the more complex the search problem.
    
    \subsubsection{Experiments}
    
    \begin{figure}[htb!]
    	\centering
    	\includegraphics[width=7cm]{PathLength.png}
    	\caption{}
    	\label{fig:1}
    \end{figure}

	\begin{figure}[htb!]
		\centering
		\includegraphics[width=7cm]{NodesExpanded.png}
		\caption{}
		\label{fig:2}
	\end{figure}

	\begin{figure}[htb!]
		\centering
		\includegraphics[width=7cm]{Memory.png}
		\caption{}
		\label{fig:3}
	\end{figure}

	\begin{figure}[htb!]
		\centering
		\includegraphics[width=7cm]{Time.png}
		\caption{}
		\label{fig:4}
	\end{figure}

	

\section{Conclusion}
In this research, our team explored the intuition, implementation, and evaluation of three search strategies for agents in partially-observable (specifically eight-connected) environments. These search strategies were integrated with the Pacman domain, which is a representative application for robot navigation search tasks.
%The metrics for evaluation were the final path length, number of nodes expanded from the search queue, memory use, and runtime.
After evaluating the three search strategies (in addition to A*), the group confirmed that the D* Lite strategy is the optimal re-planning strategy. The D* Lite builds off the theoretical foundation of LPA*, but makes signification optimizations to the implementation that allow it to dominate in terms of performance. The NRA* strategy is competitive in terms of path length, but its simple strategy hurts its performance in all other categories. When a totally observable environment is not practical, D* Lite is the best search strategy of the assessed algorithms.

\section{Team Notes}
The team split the tasks of the research into clearly defined tasks. Hans implemented the core LPA* and D* Lite algorithms, assisted with integration with Pacman domain, and helped to facilitate the distribution of tasks for other team members. Rishabh worked on integrating the core algorithms with the Pacman domain, evaluating the algorithms, and the group used his existing A* implementation. Nicholas assisted with the Pacman integration, evaluation, tabulating and graphing of results. Bharath also assisted with Pacman integration, evaluation, and debugging efforts. All team members contributed to the report.


\section*{Acknowledgment}
% better safe than sorry!
We would like to thank Mehrdad Zaker Shahrak and Yu Zhang for sharing their valuable insights for this work.

\bibliographystyle{ieeetr}
\bibliography{refs.bib}

\end{document}
